\documentclass{standalone}
\usepackage{standalone}

\begin{document}
\chapter{Introduction}
Today social media is a great source of data. Everyday people are using these sites and leaving out a huge amount of data about their current situation, what they likes, their photos, opinions, feelings, places they visited, among their information. This information much more relevant because there is no difference between real world and social world. People express themselves in real world as they express themselves in SNS. So predicting users gender using Facebook status is a best idea. Main reason for that is the huge amount of relevant data. SNS like Facebook, Twitter, Weibo etc. are popular right now. LinkedIn is already transformed themselves into some kind of SNS. Facebook is one of the most popular types of SNS. It has 1.86 billion monthly active users, as of the fourth quarter of 2016. There are about 3.7 billion internet users (about 47\% of world population) and about 2.3 billion use social media.\\
People are providing huge amount of data in their day to day life. For example, number of likes in a status, number of tag which defines how social that specific user is, number of friends, birthday, number of group managed, number of cover photo, profile pictures and albums, check-in, number of status, city name etc. These features contains a lot of information about that specific user. \\

That's why in our approach we considered users Facebook status. Status can vary from gender to gender.
There are a lot of similarity in status between male and female. But less difference between them. We tried to find dissimilarity between them with our model, test our model and showed that it is possible to classify gender according to their status.\\
Similar task was done on age features. Status can also vary from age to age. Person with different age upload different types of status. So it is also possible to classify the age level depending on their age.\\
Religion can also be classified by status. Person with different religion upload different religious status. Not every people upload religious views in social media. So sometimes it is difficult to identify their religious views. So we tried to keep balance in religious ratio. If the ratio will maintained well our model will also be performed well. \\
 
\\

\end{document}
